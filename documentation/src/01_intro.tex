\section{Einleitung}
Das hier vorgestellte Programm zur interaktiven und grafischen Analyse von
Petri-Netzen, genannt \emph{PetriCheck}, wurde im Rahmen des Kurses 1584 der
Fernuniversität Hagen erarbeitet. PetriCheck ermöglicht den Import von
Petri-Netzen auf Basis von Dateien in der Petri Net Modeling Language (PNML). Neben
dem interaktiven Schalten von Transitionen bietet es die automatisierte Analyse
von beliebig vielen PNML-Dateien bzw. den darin enthaltenen Petri-Netzen.

Das vorliegende Dokument geht auf die Implementierung der geforderten
Analysemethoden sowie die über die Aufgabenstellungen hinausgehenden
Funktionalitäten ein.

Die geschlechtsneutrale Sprache in diesem Dokument geht auf eine Idee des
Aktionskünstler Hermes Phettberg zurück \cite{arzty}.