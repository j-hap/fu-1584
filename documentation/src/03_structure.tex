\section{Programmstruktur}
\label{sec:struct}

In diesem Abschnitt wird auf die Implementierung der Modell-Klassen und die
darin verwendeten Datenstrukturen eingegangen. Daneben wird auf einige wenige
spezielle Aspekte der Implementierung von PetriCheck eingegangen.

\subsection{PetriNet}

Die Klasse \texttt{PetriNet} bildet die Modell-Repräsentation eines Petri-Netzes
in PetriCheck dar. Wegen der Einzigartigkeit der Element-IDs werden die Stellen,
Transitionen und Kanten als eigene Objekte jeweils in \texttt{Map}s
organisiert. Dies ermöglicht schnellen und komfortablen Zugriff auf die Elemente
des Netzes mittels ihrer ID.

Um sicherzustellen, dass jede ID einzigartig ist ist, wird bei jedem Hinzufügen
eines neuen Elementes geprüft, ob die ID des zu erzeugenden Elementes
einzigartig ist. Hierfür wird ein \texttt{Set} von IDs befüllt und vor jedem
Einfügen geprüft, ob die neue Element-ID in diesem Netz schon in Verwendung ist.
Wird die Einzigartigkeitsbedingung verletzt, so informiert das PetriNet die
aufrufende Methode mittels Exception darüber. In PetriCheck erfolgt der Aufbau
des Petri-Netzes ausschließlich auf Dateibasis. Der dafür verwendete
\texttt{SimplePnmlParser} kann durch Abfangen dieser Exceptions ein Anwendy über
Fehldefinitionen innerhalb der PNML-Datei über Logger-Ausgaben informieren.

Die Kanten eines Petri-Netzes werden zwar auch innerhalb des \texttt{PetriNet}
Objektes gehalten, allerdings dienen diese nur dazu eine grafische
Repräsentation der Kanten zu erstellen. Die eigentliche Schaltlogik ist auf
Elementebene implementiert. Stellen und Transitionen sind als eigene Klassen
implementiert. Jeder Transition werden beim Hinzufügen der Kanten zum Petri-Netz
ihre Vorgänger- und Nachfolger-Stellen bekannt gemacht. Das Petri-Netz selbst
gibt zum Schalten einer Transition nur der zu schaltenen Transition Bescheid.
Die Umverteilung von Tokens geschieht dann auf Element-Ebene.

\subsubsection{Transition}
Die Klasse \texttt{Transition} hält in zwei \texttt{Set}s die Vorgänger- und
Nachfolger-Stellen dieser Transition. Da laut Aufgabenstellung keine doppelten
Kanten in den Modellen zu erwarten sind, wurde das Set einer Liste vorgezogen.
Durch Prüfen aller Vorgänger-Stellen, wird ermittelt, ob eine Transition aktiv
ist. Ändert sich der Status, so werden Listener darüber definiert. In PetriCheck
ist der \texttt{PetriNetController} Listener für jedes Element des Petri-Netzes
und propagiert Änderungen, die eine geänderte Darstellung erfordern, an die
Objekte der GraphStream Bibliothek.

\subsubsection{Place}
Zur Repräsentation einer Stelle im Petri-Netz dient die \texttt{Place}-Klasse.
In einem Objekt dieser Klasse sind die aktuelle Anzahl Marken sowie die initiale
Anzahl Marken hinterlegt. Die Klasse bietet Methoden zur Inkrementierung und
Dekrementierung der aktuellen Marken, wobei die Validität (keine negative
Anzahl, kein Integer Overflow) sichergestellt wird. Desweiteren informiert eine
Objekt dieser Klasse eventuell vorhandene Listener über eine Änderung der
aktuellen Makenanzahl. Dies wird genutzt, um die Darstellung des Netzes aktuell
zu halten.

\subsection{ReachabilityGraph}
Die Klasse \texttt{ReachabilityGraph} bildet die Modell-Repräsentation eines
Erreichbarkeitsgraphen in PetriCheck.

Wegen der einfacheren Synchronisation erstellt jedes \texttt{PetriNet} Objekt
bei seiner Erstellung direkt ein \texttt{ReachabilityGraph}-Objekt. So kann
sichergestellt werden, dass jedes Petri-Netz stets einen eigenen
Erreichbarkeitsgraphen pflegt und die Schalthistorie verfolgt wird. Die Knoten
des Erreichbarkeitsgraphen werden durch \texttt{LinkedMarking}-Objekte gebildet.
Dies ist eine Subklasse der \texttt{Marking}-Klasse.

\subsubsection{Marking}
Ein \texttt{Marking}-Objekt stellt eine Markierung eines Petri-Netzes dar.
Hierbei wird die Anzahl der Tokens einer jeden Stelle durch eine einfache
Ganzzahl in einem Array repräsentiert. Die Ordnung in diesem Array entspricht
der alphabetischen Ordnung der IDs der Stellen. Zwei \texttt{Marking}s mit
derselben Markierung geben den gleichen Hash-Wert aus. Ebenso bezieht sich der
Vergleich mit der \texttt{equals}-Methode ausschließlich auf die repräsentierte
Markierung. Dies vereinfacht die Handhabung von zwei unterschiedlichen
\texttt{Marking}s, die aber die selbe Markierung des Petri-Netzes
repräsentieren. Die \texttt{compareTo}-Methode enthält die Prüfung der m
$\leftrightarrow$ m' Relation, die bei der Beschränktheitsanalyse genutzt wird.

\subsubsection{LinkedMarking}
Ein \texttt{LinkedMarking} enthält neben den geerbten Attributen des
\texttt{Marking}s die Verbindungen zu anderen \texttt{LinkedMarking}s, sowohl
Vorgänger als auch Nachfolger. Da die Kanten in einem Erreichbarkeitsgraphen
gelabelt sind, dient die generische Klasse \texttt{Edge} als Repräsentation
einer gelabelten Kante.

\subsubsection{Edge}
Jedes \texttt{Edge}-Objekt enthält den Namen der Kante sowie den Ursprungs- oder
Zielknoten der Kante, je nachdem wie sie verwendet wird. Die Nachbarn eines
Objekts vom Typ \texttt{T} können also in zwei \texttt{Set}s von
\texttt{Edge<T>}-Objekten gehalten. Dies ermöglicht die Identifikation von
Vorgänger- und Nachfolgermarkierungen, was beim Aufbau des
Erreichbarkeitsgraphen als auch bei der Beschränktheitsanalyse dienlich ist
(siehe \cref{sec:algo}).

Wie bei den \texttt{Marking}-Objekten gelten zwei Edges als gleich, wenn sie das
gleiche Label und den laut Hashwert gleichen Endknoten haben. Zwei Edge-Objekte
liefern dann also auch den gleichen Hash-Wert, der zur Identifikation in
\texttt{HashSet}s genutzt wird. Dies ermöglicht die einfache Kontrolle, ob im
Petri-Netz geschaltete Trasitionen schon im Netz vorhanden sind, auch wenn diese
Kante durch ein separates Objekt repräsentiert wird.

\subsection{UI-Actions}
Die Implementierung der dem Anwendy angebotenen Aktionen in der grafischen
Bedienoberfläche (GUI) ist mit möglichst geringer Code-Duplikation umgesetzt. Es
können einige Aktionen in der Oberfläche an mehreren Stellen ausgelöst werden,
zum Beispiel im Menüband als auch in der Toolbar. Es soll aber nur eine
Repräsentation dieser Aktion existieren. Hierfür werden die \texttt{Action}s als
Eigenschaften von \texttt{Enum}-Objekten in der \texttt{MainViewAction}
Enumeration gehalten. Jedes Objekt dieser Aufzählung wird bei der Erstellung mit
einem anzuzeigenden Befehlstext, einem Hinweistext (Tooltip) und einem Element
der \texttt{UiIcon}-Enumeration versehen. Innerhalb eines
\texttt{UiIcon}-Objekts sind Icons in zwei Größen eingebettet, ein Icon für die
Darstellung eines Toolbar-Buttons und ein Icon für die Visualisierung innerhalb
des Anwendungsmenüs. Diese Icons werden an die entsprechenden Stellen in der
Action gesetzt, sodass das Swing Framework das jeweils geeignete Icon in der
Oberfläche darstellt.

Die unterhalb der einzelnen \texttt{MainViewAction}-Einträgen liegenden
\texttt{Action}s werden beim Aufbau der Bedienoberfläche an den jeweilgen
Stellen eingesetzt. Wird nun eine dieser Action aus dem GUI heraus getriggert,
so erfolgt eine Weiterleitung der Aktion an ein zentrales Mediator-Objekt, das
im Vorfeld der \texttt{MainViewAction}-Aufzählung mitgeteilt wurde.

Bei der Weiterleitung wird allerdings der ursprüngliche Auslöser der Aktion, die
\texttt{Action} eines \texttt{MainViewAction}-Objekts, durch das
\texttt{MainViewAction}-Objekt selbst ersetzt. Dies ermöglicht eine komfortable
Entscheidung mittels \texttt{switch}-Statements darüber, welche Aktion im
Mediator-Objekt ausgelöst werden soll.

Die Umsetztung sowohl der \texttt{MainViewAction}- als auch der
\texttt{UiIcon}-Objekte in Aufzählungen erlaubt die einfache Verwaltung aller
verfügbaren Aktionen an zentraler Stelle bei geringem Code-Umfang.

Die Rolle des Mediator-Objekts nimmt die \texttt{MainController}-Klasse ein.

\subsection{Tabbed Document Interface}
Die Verwaltung der Tabs und die Assiziation der Tabs mit den geöffneren Dateien
übernimmt ein Objekt der Klasse \texttt{TabManager} bzw. ein Objekt der auf das
Swing-Framework spezialisierten Klasse \texttt{SwingTabManager}. Da keine Datei
doppelt geöffnet werden kann, verwaltet ein \texttt{SwingTabManager} die offenen
Tabs in \texttt{Map}s, um Tabs mit absoluten Dateipfaden und umgekehrt zu
assizieren. Außerdem benachritigt ein \texttt{TabManager} eventuell vorhandene
Listener-Objekte über ein Umschalten des aktiven Tabs durch das Anwendy. In
PetriCheck sind dies der genutzte \texttt{FileSelector} und der
\texttt{MainController}. Der \texttt{FileSelector} muss die aktuelle Datei
kennen, damit die Funktionen "nächste Datei" und "vorherige Datei" wie erwartet
funktionieren. Der \texttt{MainController} sorgt dafür, dass in der Statusleiste
des Hauptfenster der absolute Pfad des aktuell betrachteten Petri-Netzes
angezeigt wird.

Zusätzlich kümmert sich der \texttt{TabManager} darum, dass im globalen Logger
das Textfeld des jeweils aktiven Tab als Ausgabe genutzt wird. Der globale
Logger gibt Meldungen sowohl an die Standardausgabe aus als auch an den aktuell
sichtbaren Textbereich im Fenster.

Das Inhalts-Panel der Tabs selbst wird durch die Klasse \texttt{SwingTab}
implementiert.